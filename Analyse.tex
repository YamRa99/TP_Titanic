% Options for packages loaded elsewhere
\PassOptionsToPackage{unicode}{hyperref}
\PassOptionsToPackage{hyphens}{url}
%
\documentclass[
]{article}
\usepackage{amsmath,amssymb}
\usepackage{iftex}
\ifPDFTeX
  \usepackage[T1]{fontenc}
  \usepackage[utf8]{inputenc}
  \usepackage{textcomp} % provide euro and other symbols
\else % if luatex or xetex
  \usepackage{unicode-math} % this also loads fontspec
  \defaultfontfeatures{Scale=MatchLowercase}
  \defaultfontfeatures[\rmfamily]{Ligatures=TeX,Scale=1}
\fi
\usepackage{lmodern}
\ifPDFTeX\else
  % xetex/luatex font selection
\fi
% Use upquote if available, for straight quotes in verbatim environments
\IfFileExists{upquote.sty}{\usepackage{upquote}}{}
\IfFileExists{microtype.sty}{% use microtype if available
  \usepackage[]{microtype}
  \UseMicrotypeSet[protrusion]{basicmath} % disable protrusion for tt fonts
}{}
\makeatletter
\@ifundefined{KOMAClassName}{% if non-KOMA class
  \IfFileExists{parskip.sty}{%
    \usepackage{parskip}
  }{% else
    \setlength{\parindent}{0pt}
    \setlength{\parskip}{6pt plus 2pt minus 1pt}}
}{% if KOMA class
  \KOMAoptions{parskip=half}}
\makeatother
\usepackage{xcolor}
\usepackage[margin=1in]{geometry}
\usepackage{color}
\usepackage{fancyvrb}
\newcommand{\VerbBar}{|}
\newcommand{\VERB}{\Verb[commandchars=\\\{\}]}
\DefineVerbatimEnvironment{Highlighting}{Verbatim}{commandchars=\\\{\}}
% Add ',fontsize=\small' for more characters per line
\usepackage{framed}
\definecolor{shadecolor}{RGB}{248,248,248}
\newenvironment{Shaded}{\begin{snugshade}}{\end{snugshade}}
\newcommand{\AlertTok}[1]{\textcolor[rgb]{0.94,0.16,0.16}{#1}}
\newcommand{\AnnotationTok}[1]{\textcolor[rgb]{0.56,0.35,0.01}{\textbf{\textit{#1}}}}
\newcommand{\AttributeTok}[1]{\textcolor[rgb]{0.13,0.29,0.53}{#1}}
\newcommand{\BaseNTok}[1]{\textcolor[rgb]{0.00,0.00,0.81}{#1}}
\newcommand{\BuiltInTok}[1]{#1}
\newcommand{\CharTok}[1]{\textcolor[rgb]{0.31,0.60,0.02}{#1}}
\newcommand{\CommentTok}[1]{\textcolor[rgb]{0.56,0.35,0.01}{\textit{#1}}}
\newcommand{\CommentVarTok}[1]{\textcolor[rgb]{0.56,0.35,0.01}{\textbf{\textit{#1}}}}
\newcommand{\ConstantTok}[1]{\textcolor[rgb]{0.56,0.35,0.01}{#1}}
\newcommand{\ControlFlowTok}[1]{\textcolor[rgb]{0.13,0.29,0.53}{\textbf{#1}}}
\newcommand{\DataTypeTok}[1]{\textcolor[rgb]{0.13,0.29,0.53}{#1}}
\newcommand{\DecValTok}[1]{\textcolor[rgb]{0.00,0.00,0.81}{#1}}
\newcommand{\DocumentationTok}[1]{\textcolor[rgb]{0.56,0.35,0.01}{\textbf{\textit{#1}}}}
\newcommand{\ErrorTok}[1]{\textcolor[rgb]{0.64,0.00,0.00}{\textbf{#1}}}
\newcommand{\ExtensionTok}[1]{#1}
\newcommand{\FloatTok}[1]{\textcolor[rgb]{0.00,0.00,0.81}{#1}}
\newcommand{\FunctionTok}[1]{\textcolor[rgb]{0.13,0.29,0.53}{\textbf{#1}}}
\newcommand{\ImportTok}[1]{#1}
\newcommand{\InformationTok}[1]{\textcolor[rgb]{0.56,0.35,0.01}{\textbf{\textit{#1}}}}
\newcommand{\KeywordTok}[1]{\textcolor[rgb]{0.13,0.29,0.53}{\textbf{#1}}}
\newcommand{\NormalTok}[1]{#1}
\newcommand{\OperatorTok}[1]{\textcolor[rgb]{0.81,0.36,0.00}{\textbf{#1}}}
\newcommand{\OtherTok}[1]{\textcolor[rgb]{0.56,0.35,0.01}{#1}}
\newcommand{\PreprocessorTok}[1]{\textcolor[rgb]{0.56,0.35,0.01}{\textit{#1}}}
\newcommand{\RegionMarkerTok}[1]{#1}
\newcommand{\SpecialCharTok}[1]{\textcolor[rgb]{0.81,0.36,0.00}{\textbf{#1}}}
\newcommand{\SpecialStringTok}[1]{\textcolor[rgb]{0.31,0.60,0.02}{#1}}
\newcommand{\StringTok}[1]{\textcolor[rgb]{0.31,0.60,0.02}{#1}}
\newcommand{\VariableTok}[1]{\textcolor[rgb]{0.00,0.00,0.00}{#1}}
\newcommand{\VerbatimStringTok}[1]{\textcolor[rgb]{0.31,0.60,0.02}{#1}}
\newcommand{\WarningTok}[1]{\textcolor[rgb]{0.56,0.35,0.01}{\textbf{\textit{#1}}}}
\usepackage{graphicx}
\makeatletter
\def\maxwidth{\ifdim\Gin@nat@width>\linewidth\linewidth\else\Gin@nat@width\fi}
\def\maxheight{\ifdim\Gin@nat@height>\textheight\textheight\else\Gin@nat@height\fi}
\makeatother
% Scale images if necessary, so that they will not overflow the page
% margins by default, and it is still possible to overwrite the defaults
% using explicit options in \includegraphics[width, height, ...]{}
\setkeys{Gin}{width=\maxwidth,height=\maxheight,keepaspectratio}
% Set default figure placement to htbp
\makeatletter
\def\fps@figure{htbp}
\makeatother
\setlength{\emergencystretch}{3em} % prevent overfull lines
\providecommand{\tightlist}{%
  \setlength{\itemsep}{0pt}\setlength{\parskip}{0pt}}
\setcounter{secnumdepth}{-\maxdimen} % remove section numbering
\ifLuaTeX
  \usepackage{selnolig}  % disable illegal ligatures
\fi
\usepackage{bookmark}
\IfFileExists{xurl.sty}{\usepackage{xurl}}{} % add URL line breaks if available
\urlstyle{same}
\hypersetup{
  pdftitle={Analyse},
  hidelinks,
  pdfcreator={LaTeX via pandoc}}

\title{Analyse}
\author{}
\date{\vspace{-2.5em}2024-11-14}

\begin{document}
\maketitle

\subsection{}\label{section}

\begin{Shaded}
\begin{Highlighting}[]
\CommentTok{\#Charger les données : }
\FunctionTok{library}\NormalTok{(carData)}
\end{Highlighting}
\end{Shaded}

\begin{verbatim}
## Warning: le package 'carData' a été compilé avec la version R 4.4.2
\end{verbatim}

\begin{Shaded}
\begin{Highlighting}[]
\NormalTok{BDD }\OtherTok{\textless{}{-}}\NormalTok{ TitanicSurvival}
\end{Highlighting}
\end{Shaded}

Présentation des données : Le data frame ``BDD'' contient 1309
observations avec 4 variables :

1- survived : survie du passager (yes ou no), variable catégorielle
binaire.

2- sex : sexe du passager (female ou male), variable catégorielle
nominale.

3- age : âge en années (avec 263 valeurs manquantes), variable numérique
continue.

4- passengerClass : classe du billet (1st, 2nd, 3rd), variable
catégorielle ordinale.

\begin{Shaded}
\begin{Highlighting}[]
\CommentTok{\#Présentez les statistiques descriptives}
\FunctionTok{summary}\NormalTok{(BDD)}
\end{Highlighting}
\end{Shaded}

\begin{verbatim}
##  survived      sex           age          passengerClass
##  no :809   female:466   Min.   : 0.1667   1st:323       
##  yes:500   male  :843   1st Qu.:21.0000   2nd:277       
##                         Median :28.0000   3rd:709       
##                         Mean   :29.8811                 
##                         3rd Qu.:39.0000                 
##                         Max.   :80.0000                 
##                         NA's   :263
\end{verbatim}

\#Graphique pour la répartition des survivants en fonction du sexe avec
des couleurs personnalisées

\begin{Shaded}
\begin{Highlighting}[]
\FunctionTok{library}\NormalTok{(ggplot2)}
\end{Highlighting}
\end{Shaded}

\begin{verbatim}
## Warning: le package 'ggplot2' a été compilé avec la version R 4.4.2
\end{verbatim}

\begin{Shaded}
\begin{Highlighting}[]
\FunctionTok{ggplot}\NormalTok{(BDD, }\FunctionTok{aes}\NormalTok{(}\AttributeTok{x =}\NormalTok{ sex, }\AttributeTok{fill =}\NormalTok{ survived)) }\SpecialCharTok{+}
  \FunctionTok{geom\_bar}\NormalTok{(}\AttributeTok{position =} \StringTok{"dodge"}\NormalTok{) }\SpecialCharTok{+}
  \FunctionTok{labs}\NormalTok{(}\AttributeTok{title =} \StringTok{"Répartition des survivants en fonction du sexe"}\NormalTok{,}
       \AttributeTok{x =} \StringTok{"Sexe"}\NormalTok{,}
       \AttributeTok{y =} \StringTok{"Nombre"}\NormalTok{,}
       \AttributeTok{fill =} \StringTok{"Survie"}\NormalTok{) }\SpecialCharTok{+}
  \FunctionTok{scale\_fill\_manual}\NormalTok{(}\AttributeTok{values =} \FunctionTok{c}\NormalTok{(}\StringTok{"orange"}\NormalTok{, }\StringTok{"blue"}\NormalTok{)) }\SpecialCharTok{+}
  \FunctionTok{theme\_minimal}\NormalTok{()}
\end{Highlighting}
\end{Shaded}

\includegraphics{Analyse_files/figure-latex/unnamed-chunk-3-1.pdf}

Interprétation : Le graphique montre que le nombre de non-survivants (en
orange) est supérieur chez les hommes par rapport aux femmes. À
l'inverse, le nombre de survivants (en bleu) est nettement plus élevé
chez les femmes, soulignant une différence significative entre les sexes
en termes de survie.

\#Graphique : Répartition des survivants en fonction de la classe

\begin{Shaded}
\begin{Highlighting}[]
\FunctionTok{ggplot}\NormalTok{(BDD, }\FunctionTok{aes}\NormalTok{(}\AttributeTok{x =}\NormalTok{ passengerClass, }\AttributeTok{fill =}\NormalTok{ survived)) }\SpecialCharTok{+}
  \FunctionTok{geom\_bar}\NormalTok{(}\AttributeTok{position =} \StringTok{"dodge"}\NormalTok{) }\SpecialCharTok{+}
  \FunctionTok{labs}\NormalTok{(}\AttributeTok{title =} \StringTok{"Répartition des survivants en fonction de la classe"}\NormalTok{,}
       \AttributeTok{x =} \StringTok{"Classe"}\NormalTok{,}
       \AttributeTok{y =} \StringTok{"Nombre"}\NormalTok{,}
       \AttributeTok{fill =} \StringTok{"Survie"}\NormalTok{) }\SpecialCharTok{+}
   \FunctionTok{scale\_fill\_manual}\NormalTok{(}\AttributeTok{values =} \FunctionTok{c}\NormalTok{(}\StringTok{"purple"}\NormalTok{, }\StringTok{"green"}\NormalTok{)) }\SpecialCharTok{+}
  \FunctionTok{theme\_minimal}\NormalTok{()}
\end{Highlighting}
\end{Shaded}

\includegraphics{Analyse_files/figure-latex/unnamed-chunk-4-1.pdf}

Interprétation : Le graphique montre que les survivants (en vert) sont
principalement issus de la première classe, tandis que les
non-survivants (en violet) proviennent principalement de la troisième
classe, ce qui indique une disparité liée au statut socio-économique

\#Représentation de l'age en fonction de la variable survived.

\begin{Shaded}
\begin{Highlighting}[]
\FunctionTok{ggplot}\NormalTok{(BDD, }\FunctionTok{aes}\NormalTok{(}\AttributeTok{x =} \FunctionTok{as.factor}\NormalTok{(survived), }\AttributeTok{y =}\NormalTok{ age, }\AttributeTok{fill =} \FunctionTok{as.factor}\NormalTok{(survived))) }\SpecialCharTok{+}
  \FunctionTok{geom\_boxplot}\NormalTok{() }\SpecialCharTok{+}
  \FunctionTok{labs}\NormalTok{(}\AttributeTok{title =} \StringTok{"Répartition de l\textquotesingle{}âge en fonction de la survie"}\NormalTok{,}
       \AttributeTok{x =} \StringTok{"Survie (0 = Non, 1 = Oui)"}\NormalTok{,}
       \AttributeTok{y =} \StringTok{"Âge"}\NormalTok{,}
       \AttributeTok{fill =} \StringTok{"Survécu"}\NormalTok{) }\SpecialCharTok{+}
  \FunctionTok{scale\_fill\_manual}\NormalTok{(}\AttributeTok{values =} \FunctionTok{c}\NormalTok{(}\StringTok{"pink"}\NormalTok{, }\StringTok{"skyblue"}\NormalTok{)) }\SpecialCharTok{+}
  \FunctionTok{theme\_minimal}\NormalTok{()}
\end{Highlighting}
\end{Shaded}

\begin{verbatim}
## Warning: Removed 263 rows containing non-finite outside the scale range
## (`stat_boxplot()`).
\end{verbatim}

\includegraphics{Analyse_files/figure-latex/unnamed-chunk-5-1.pdf}

Interprétation : Le graphique montre la répartition des survivants et
des non-survivants selon l'âge, sans différence significative entre les
groupes.

\section{Chargement du package
``TitanicPckg''}\label{chargement-du-package-titanicpckg}

\begin{Shaded}
\begin{Highlighting}[]
\FunctionTok{library}\NormalTok{(TitanicPckg)}

\FunctionTok{taux\_survie\_class}\NormalTok{(BDD)}
\end{Highlighting}
\end{Shaded}

\begin{verbatim}
## 
## Attachement du package : 'dplyr'
\end{verbatim}

\begin{verbatim}
## Les objets suivants sont masqués depuis 'package:stats':
## 
##     filter, lag
\end{verbatim}

\begin{verbatim}
## Les objets suivants sont masqués depuis 'package:base':
## 
##     intersect, setdiff, setequal, union
\end{verbatim}

\begin{verbatim}
## # A tibble: 3 x 2
##   passengerClass TauxSurvie
##   <fct>               <dbl>
## 1 1st                 0.619
## 2 2nd                 0.430
## 3 3rd                 0.255
\end{verbatim}

\begin{Shaded}
\begin{Highlighting}[]
\FunctionTok{taux\_survie\_sex}\NormalTok{(BDD)}
\end{Highlighting}
\end{Shaded}

\begin{verbatim}
## # A tibble: 2 x 2
##   sex    TauxSurvie
##   <fct>       <dbl>
## 1 female      0.727
## 2 male        0.191
\end{verbatim}

\end{document}
